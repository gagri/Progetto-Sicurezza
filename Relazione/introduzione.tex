\chapter{Introduzione}

Android è, ad oggi, il sistema operativo più diffuso nei dispositivi portatili e sta aumentando l'utilizzo dello stesso in vari elettrodomestici intelligenti. Data la forte dipendenza da sistemi di tale genere, la presenza di security issue può avere conseguenza anche gravi. Eppure, regolarmente, incidenti di sicurezza di alto profilo con la piattaforma Android mettono in luce la facilità con cui gli autori di malware possono sfruttare una grande superficie di attacco, eludendo tutti i sistemi di rilevamento sia a livello di app store che a livello di dispositivo. Ciò porta a dedurre che gli approcci allo stato dell'arte non sono sufficientemente maturi per affrontare in maniera efficace il malware e che gli autori di malware sono ancora in grado di reagire rapidamente alle capacità delle attuali tecniche di rilevamento.

Le applicazioni Android vengono distribuite come pacchetti software che includono il bytecode, le risorse e il Manifest il quale contiene le informazioni essenziali relative all'app stessa. Tra queste informazioni si hanno i permessi necessari e la lista dei componenti che il sistema deve conoscere per poter eseguire il codice dell'app. Sfortunatamente, a differenza dei tradizionali file eseguibili, i package di un'app Android possono essere facilmente modificati da terze parti. I writer di malware sfruttano infatti la popolarità di alcune app per diffondere il codice malevolo. Infatti, è abbastanza semplice effettuare l'unpack di un'app, preferibilmente popolare, inserire al suo interno codice malevolo al suo interno e quindi ridistribuirla gratuitamente. L'app risultante, la quale trasporta codice malevolo, è nota come \textbf{piggybacked app}. In contrasto con il comune repackaging, in cui il codice dell'app originale non viene modificato, nel caso delle \textbf{piggybacked app}, viene aggiunto codice extra affinché l'app abbia un comportamento diverso, generalmente malevolo, rispetto a quello dell'app originale. 
Il repackiging non necessariamente richiede la modifica del bytecode di una data app in quanto è possibile effettuare il repakiging dopo aver semplicemente modifcato l'owner. Piggybacking è un'app android ripacchettizzata dopo aver apportato modifiche al contenuto dell'app stessa, ad esempio inserendo payload malevolo, pubblicità o altro. Si tratta quindi di un sottoinsieme delle app ripacchettizzate.
Risulta quindi interessante studiare tali app al fine di comprendere la creazione e distribuzione del malware stesso.
Il progetto in esame parte da un dataset costituito da circa INSERIRE UN NUMERO coppie di app (originale, piggybacked). L'obiettivo è quello di analizzare le caratteristiche strutturali di app piggibacked relazionando le stesse con l'attivazione del malware. 
MODIFICARE PARTE CON OBIETTIVI 
I parametri presi in considerazione sono i seguenti:
\begin{itemize}
	\item permessi
	\item Activity aggiunte
	\item caricamento dinamico
	\item dipendenze 
\end{itemize}
In \autoref{fig:intro1} si mostrano le parti costitutive di un'app piggybacked. Nella maggior parte dei casi in cui è presente malware, l'app originale (definita in letteratura \textbf{carrier}) viene modificata aggiungendo payload malevolo (noto in letteratura come \textbf{rider}). Il comportamento malevolo può essere attivato grazie al cosiddetto \textbf{hook code} che viene inserito dall'autore del malware per connettere \textbf{rider} e \textbf{carrier}. L'\textbf{hook} fa quindi in modo che il controllo passi dal \textbf{carrier} al \textbf{rider}.
\section{Carrier, hook e rider}
\begin{figure}[h]
	\centering
	\includegraphics[width=0.7\linewidth]{fig/app.JPG}
	\caption{Partite costituenti piggybacked app}
	\label{fig:intro1}
\end{figure}
